% Options for packages loaded elsewhere
\PassOptionsToPackage{unicode}{hyperref}
\PassOptionsToPackage{hyphens}{url}
%
\documentclass[
]{article}
\usepackage{amsmath,amssymb}
\usepackage{lmodern}
\usepackage{iftex}
\ifPDFTeX
  \usepackage[T1]{fontenc}
  \usepackage[utf8]{inputenc}
  \usepackage{textcomp} % provide euro and other symbols
\else % if luatex or xetex
  \usepackage{unicode-math}
  \defaultfontfeatures{Scale=MatchLowercase}
  \defaultfontfeatures[\rmfamily]{Ligatures=TeX,Scale=1}
\fi
% Use upquote if available, for straight quotes in verbatim environments
\IfFileExists{upquote.sty}{\usepackage{upquote}}{}
\IfFileExists{microtype.sty}{% use microtype if available
  \usepackage[]{microtype}
  \UseMicrotypeSet[protrusion]{basicmath} % disable protrusion for tt fonts
}{}
\makeatletter
\@ifundefined{KOMAClassName}{% if non-KOMA class
  \IfFileExists{parskip.sty}{%
    \usepackage{parskip}
  }{% else
    \setlength{\parindent}{0pt}
    \setlength{\parskip}{6pt plus 2pt minus 1pt}}
}{% if KOMA class
  \KOMAoptions{parskip=half}}
\makeatother
\usepackage{xcolor}
\usepackage[margin=1in]{geometry}
\usepackage{color}
\usepackage{fancyvrb}
\newcommand{\VerbBar}{|}
\newcommand{\VERB}{\Verb[commandchars=\\\{\}]}
\DefineVerbatimEnvironment{Highlighting}{Verbatim}{commandchars=\\\{\}}
% Add ',fontsize=\small' for more characters per line
\usepackage{framed}
\definecolor{shadecolor}{RGB}{248,248,248}
\newenvironment{Shaded}{\begin{snugshade}}{\end{snugshade}}
\newcommand{\AlertTok}[1]{\textcolor[rgb]{0.94,0.16,0.16}{#1}}
\newcommand{\AnnotationTok}[1]{\textcolor[rgb]{0.56,0.35,0.01}{\textbf{\textit{#1}}}}
\newcommand{\AttributeTok}[1]{\textcolor[rgb]{0.77,0.63,0.00}{#1}}
\newcommand{\BaseNTok}[1]{\textcolor[rgb]{0.00,0.00,0.81}{#1}}
\newcommand{\BuiltInTok}[1]{#1}
\newcommand{\CharTok}[1]{\textcolor[rgb]{0.31,0.60,0.02}{#1}}
\newcommand{\CommentTok}[1]{\textcolor[rgb]{0.56,0.35,0.01}{\textit{#1}}}
\newcommand{\CommentVarTok}[1]{\textcolor[rgb]{0.56,0.35,0.01}{\textbf{\textit{#1}}}}
\newcommand{\ConstantTok}[1]{\textcolor[rgb]{0.00,0.00,0.00}{#1}}
\newcommand{\ControlFlowTok}[1]{\textcolor[rgb]{0.13,0.29,0.53}{\textbf{#1}}}
\newcommand{\DataTypeTok}[1]{\textcolor[rgb]{0.13,0.29,0.53}{#1}}
\newcommand{\DecValTok}[1]{\textcolor[rgb]{0.00,0.00,0.81}{#1}}
\newcommand{\DocumentationTok}[1]{\textcolor[rgb]{0.56,0.35,0.01}{\textbf{\textit{#1}}}}
\newcommand{\ErrorTok}[1]{\textcolor[rgb]{0.64,0.00,0.00}{\textbf{#1}}}
\newcommand{\ExtensionTok}[1]{#1}
\newcommand{\FloatTok}[1]{\textcolor[rgb]{0.00,0.00,0.81}{#1}}
\newcommand{\FunctionTok}[1]{\textcolor[rgb]{0.00,0.00,0.00}{#1}}
\newcommand{\ImportTok}[1]{#1}
\newcommand{\InformationTok}[1]{\textcolor[rgb]{0.56,0.35,0.01}{\textbf{\textit{#1}}}}
\newcommand{\KeywordTok}[1]{\textcolor[rgb]{0.13,0.29,0.53}{\textbf{#1}}}
\newcommand{\NormalTok}[1]{#1}
\newcommand{\OperatorTok}[1]{\textcolor[rgb]{0.81,0.36,0.00}{\textbf{#1}}}
\newcommand{\OtherTok}[1]{\textcolor[rgb]{0.56,0.35,0.01}{#1}}
\newcommand{\PreprocessorTok}[1]{\textcolor[rgb]{0.56,0.35,0.01}{\textit{#1}}}
\newcommand{\RegionMarkerTok}[1]{#1}
\newcommand{\SpecialCharTok}[1]{\textcolor[rgb]{0.00,0.00,0.00}{#1}}
\newcommand{\SpecialStringTok}[1]{\textcolor[rgb]{0.31,0.60,0.02}{#1}}
\newcommand{\StringTok}[1]{\textcolor[rgb]{0.31,0.60,0.02}{#1}}
\newcommand{\VariableTok}[1]{\textcolor[rgb]{0.00,0.00,0.00}{#1}}
\newcommand{\VerbatimStringTok}[1]{\textcolor[rgb]{0.31,0.60,0.02}{#1}}
\newcommand{\WarningTok}[1]{\textcolor[rgb]{0.56,0.35,0.01}{\textbf{\textit{#1}}}}
\usepackage{graphicx}
\makeatletter
\def\maxwidth{\ifdim\Gin@nat@width>\linewidth\linewidth\else\Gin@nat@width\fi}
\def\maxheight{\ifdim\Gin@nat@height>\textheight\textheight\else\Gin@nat@height\fi}
\makeatother
% Scale images if necessary, so that they will not overflow the page
% margins by default, and it is still possible to overwrite the defaults
% using explicit options in \includegraphics[width, height, ...]{}
\setkeys{Gin}{width=\maxwidth,height=\maxheight,keepaspectratio}
% Set default figure placement to htbp
\makeatletter
\def\fps@figure{htbp}
\makeatother
\setlength{\emergencystretch}{3em} % prevent overfull lines
\providecommand{\tightlist}{%
  \setlength{\itemsep}{0pt}\setlength{\parskip}{0pt}}
\setcounter{secnumdepth}{-\maxdimen} % remove section numbering
\ifLuaTeX
  \usepackage{selnolig}  % disable illegal ligatures
\fi
\IfFileExists{bookmark.sty}{\usepackage{bookmark}}{\usepackage{hyperref}}
\IfFileExists{xurl.sty}{\usepackage{xurl}}{} % add URL line breaks if available
\urlstyle{same} % disable monospaced font for URLs
\hypersetup{
  pdftitle={R Foundations},
  pdfauthor={Adejumo Ridwan Suleiman},
  hidelinks,
  pdfcreator={LaTeX via pandoc}}

\title{R Foundations}
\author{Adejumo Ridwan Suleiman}
\date{2022-12-07}

\begin{document}
\maketitle

\hypertarget{why-r}{%
\section{Why R?}\label{why-r}}

\hypertarget{installing-r-and-r-studio}{%
\section{Installing R and R Studio}\label{installing-r-and-r-studio}}

\hypertarget{setting-up-and-customizing-r-studio-ide}{%
\section{Setting Up and Customizing R Studio
IDE}\label{setting-up-and-customizing-r-studio-ide}}

\hypertarget{file-and-project-structure}{%
\section{File and Project Structure}\label{file-and-project-structure}}

\hypertarget{r-variables}{%
\section{R Variables}\label{r-variables}}

\begin{itemize}
\tightlist
\item
  You can take variables as containers for storing values of data.\\
\item
  Variables are created using the assignment operator
  \texttt{\textless{}-}.\\
\item
  Example
\end{itemize}

\begin{Shaded}
\begin{Highlighting}[]
\NormalTok{country }\OtherTok{\textless{}{-}} \StringTok{"Nigeria"}
\NormalTok{age }\OtherTok{\textless{}{-}} \DecValTok{62}
\end{Highlighting}
\end{Shaded}

\begin{itemize}
\tightlist
\item
  Text values are surrounded by double or single quotes while numeric
  not surrounded by any quotes.\\
\item
  The values of variables are called by calling the name of the
  variables.
\end{itemize}

\begin{Shaded}
\begin{Highlighting}[]
\NormalTok{country}
\end{Highlighting}
\end{Shaded}

\begin{verbatim}
## [1] "Nigeria"
\end{verbatim}

\begin{Shaded}
\begin{Highlighting}[]
\NormalTok{age}
\end{Highlighting}
\end{Shaded}

\begin{verbatim}
## [1] 62
\end{verbatim}

\begin{itemize}
\tightlist
\item
  You can also print variables with the print function in R.
\end{itemize}

\begin{Shaded}
\begin{Highlighting}[]
\FunctionTok{print}\NormalTok{(country)}
\end{Highlighting}
\end{Shaded}

\begin{verbatim}
## [1] "Nigeria"
\end{verbatim}

\begin{Shaded}
\begin{Highlighting}[]
\FunctionTok{print}\NormalTok{(age)}
\end{Highlighting}
\end{Shaded}

\begin{verbatim}
## [1] 62
\end{verbatim}

\begin{itemize}
\tightlist
\item
  The \texttt{c()} also known as the combine function lets you assign
  two or more values to a variable in R.
\end{itemize}

\begin{Shaded}
\begin{Highlighting}[]
\NormalTok{vowel\_sounds }\OtherTok{\textless{}{-}} \FunctionTok{c}\NormalTok{(}\StringTok{"a"}\NormalTok{,}\StringTok{"e"}\NormalTok{,}\StringTok{"i"}\NormalTok{,}\StringTok{"o"}\NormalTok{,}\StringTok{"u"}\NormalTok{)}
\NormalTok{prime\_numbers }\OtherTok{\textless{}{-}} \FunctionTok{c}\NormalTok{(}\DecValTok{1}\NormalTok{,}\DecValTok{3}\NormalTok{,}\DecValTok{5}\NormalTok{,}\DecValTok{11}\NormalTok{)}
\end{Highlighting}
\end{Shaded}

\begin{itemize}
\tightlist
\item
  You can assign multiple variables in R.
\end{itemize}

\begin{Shaded}
\begin{Highlighting}[]
\NormalTok{State }\OtherTok{\textless{}{-}}\NormalTok{ Capital }\OtherTok{\textless{}{-}} \StringTok{"Kano"}
\NormalTok{State}
\end{Highlighting}
\end{Shaded}

\begin{verbatim}
## [1] "Kano"
\end{verbatim}

\begin{Shaded}
\begin{Highlighting}[]
\NormalTok{Capital}
\end{Highlighting}
\end{Shaded}

\begin{verbatim}
## [1] "Kano"
\end{verbatim}

\hypertarget{variable-naming}{%
\subsection{Variable Naming}\label{variable-naming}}

You can name a variable any name you want buy make sure it is meaningful
and descriptive of the value you are assigning to the variable and take
the following into considerations when naming variables.\\
- A variable must start with characters\texttt{(a-z)} and can be a
combination of characters\texttt{(a-z)},underscores\texttt{(\_)},digits
and periods\texttt{(.)}.\\
- A variable can not start with digits or underscores\texttt{(\_)}.\\
- A variable can not start with a period followed by a
digit\texttt{(.4).\ \ \ -\ Variables\ in\ R\ are\ case\ sensitive\ i.e}Age\texttt{and}age\texttt{are\ not\ the\ same\ variable.\ -\ Special\ words\ in\ R\ such\ as}(TRUE,
FALSE, NULL,if,else,while,for)` and so on can't be used as variables.

\begin{Shaded}
\begin{Highlighting}[]
\CommentTok{\#Valid Variable Names}
\NormalTok{firstcountry }\OtherTok{=} \StringTok{"Nigeria"}
\NormalTok{second\_country }\OtherTok{=} \StringTok{"Congo"}
\NormalTok{thirdCountry }\OtherTok{=} \StringTok{"Sudan"}
\NormalTok{FOURTHCOUNTRY }\OtherTok{=} \StringTok{"Somalia"}
\NormalTok{country5 }\OtherTok{=} \StringTok{"South Africa"}
\NormalTok{country}\FloatTok{.6} \OtherTok{=} \StringTok{"Algeria"}
\NormalTok{.country7 }\OtherTok{=} \StringTok{"Kenya"}

\CommentTok{\#Invalid Variable Names}
\CommentTok{\#8variable \textless{}{-} "Morocco"}
\CommentTok{\#variable{-}9 \textless{}{-} "Tunisia"}
\CommentTok{\#variable 10 \textless{}{-} "Egypt"}
\CommentTok{\#\_variable\_11 \textless{}{-} "Rwanda"}
\CommentTok{\#variable@12 \textless{}{-} "Madagascar"}
\CommentTok{\#FALSE \textless{}{-} "Ghana"}
\end{Highlighting}
\end{Shaded}

\hypertarget{arithmetic-and-logical-operations-in-r}{%
\section{Arithmetic and Logical Operations in
R}\label{arithmetic-and-logical-operations-in-r}}

\hypertarget{arithmetic-operators}{%
\subsection{Arithmetic Operators}\label{arithmetic-operators}}

\begin{itemize}
\tightlist
\item
  addition\texttt{(+)}\\
\item
  subtraction \texttt{(-)}\\
\item
  multiplication\texttt{(*)}\\
\item
  division\texttt{(/)}\\
\item
  exponent\texttt{(\^{})}\\
\item
  Integer division\texttt{(\%/\%)}\\
\item
  Remainder division\texttt{(\%\%)}
\end{itemize}

\begin{Shaded}
\begin{Highlighting}[]
\DecValTok{4} \SpecialCharTok{+} \DecValTok{5}
\end{Highlighting}
\end{Shaded}

\begin{verbatim}
## [1] 9
\end{verbatim}

\begin{Shaded}
\begin{Highlighting}[]
\NormalTok{a }\OtherTok{=} \DecValTok{8}
\NormalTok{b }\OtherTok{=} \DecValTok{9}

\NormalTok{z }\OtherTok{=}\NormalTok{ a }\SpecialCharTok{+}\NormalTok{ b}

\NormalTok{y }\OtherTok{\textless{}{-}}\NormalTok{ x }\OtherTok{\textless{}{-}}\NormalTok{ z}

\NormalTok{u }\OtherTok{=}\NormalTok{ (a }\SpecialCharTok{+}\NormalTok{ b)}\SpecialCharTok{*}\NormalTok{(y}\SpecialCharTok{/}\NormalTok{x) }\SpecialCharTok{+}\NormalTok{ z}\SpecialCharTok{\^{}}\DecValTok{5} \SpecialCharTok{{-}} \DecValTok{10000}

\NormalTok{u}
\end{Highlighting}
\end{Shaded}

\begin{verbatim}
## [1] 1409874
\end{verbatim}

\begin{Shaded}
\begin{Highlighting}[]
\DecValTok{8}\SpecialCharTok{\%\%}\DecValTok{2}
\end{Highlighting}
\end{Shaded}

\begin{verbatim}
## [1] 0
\end{verbatim}

\begin{Shaded}
\begin{Highlighting}[]
\DecValTok{9}\SpecialCharTok{\%\%}\DecValTok{2}
\end{Highlighting}
\end{Shaded}

\begin{verbatim}
## [1] 1
\end{verbatim}

\begin{Shaded}
\begin{Highlighting}[]
\DecValTok{8}\SpecialCharTok{\%/\%}\DecValTok{2}
\end{Highlighting}
\end{Shaded}

\begin{verbatim}
## [1] 4
\end{verbatim}

\begin{Shaded}
\begin{Highlighting}[]
\DecValTok{9}\SpecialCharTok{\%/\%}\DecValTok{2}
\end{Highlighting}
\end{Shaded}

\begin{verbatim}
## [1] 4
\end{verbatim}

\hypertarget{comparison-operators}{%
\subsection{Comparison Operators}\label{comparison-operators}}

\begin{itemize}
\tightlist
\item
  Equal to\texttt{(==)}\\
\item
  Not equal to \texttt{(!=)}\\
\item
  Greater than\texttt{(\textgreater{})}\\
\item
  Less than \texttt{(\textless{})}\\
\item
  Less than or equal to \texttt{(\textless{}=)}\\
\item
  Greater than or equal to \texttt{(\textgreater{}=)}
\end{itemize}

\begin{Shaded}
\begin{Highlighting}[]
\NormalTok{(}\DecValTok{5}\SpecialCharTok{*}\DecValTok{4}\SpecialCharTok{/}\DecValTok{3}\NormalTok{) }\SpecialCharTok{\textgreater{}}\NormalTok{ (}\DecValTok{4}\SpecialCharTok{/}\DecValTok{1}\SpecialCharTok{\^{}}\FloatTok{0.5}\NormalTok{)}
\end{Highlighting}
\end{Shaded}

\begin{verbatim}
## [1] TRUE
\end{verbatim}

\begin{Shaded}
\begin{Highlighting}[]
\NormalTok{a }\OtherTok{=} \DecValTok{5}
\NormalTok{b }\OtherTok{\textless{}{-}}\NormalTok{ a }\SpecialCharTok{\textgreater{}=}\NormalTok{ (}\FloatTok{4.5{-}11.5}\NormalTok{)}
\NormalTok{b }\SpecialCharTok{==} \ConstantTok{FALSE}
\end{Highlighting}
\end{Shaded}

\begin{verbatim}
## [1] FALSE
\end{verbatim}

\begin{Shaded}
\begin{Highlighting}[]
\NormalTok{((}\DecValTok{2}\SpecialCharTok{*}\DecValTok{3}\NormalTok{)}\SpecialCharTok{\textgreater{}}\NormalTok{(}\DecValTok{5}\SpecialCharTok{/}\DecValTok{2}\NormalTok{)) }\SpecialCharTok{==}\NormalTok{ ((}\DecValTok{4{-}4}\NormalTok{)}\SpecialCharTok{\textgreater{}}\NormalTok{(}\DecValTok{0}\SpecialCharTok{\^{}{-}}\DecValTok{4}\NormalTok{))}
\end{Highlighting}
\end{Shaded}

\begin{verbatim}
## [1] FALSE
\end{verbatim}

\hypertarget{logical-operators}{%
\subsection{Logical Operators}\label{logical-operators}}

\begin{itemize}
\tightlist
\item
  AND\texttt{(\&)}\\
\item
  OR\texttt{(\textbar{})}\\
\item
  NOT\texttt{(!)}
\end{itemize}

\begin{Shaded}
\begin{Highlighting}[]
\NormalTok{(}\DecValTok{4} \SpecialCharTok{\textgreater{}} \DecValTok{5}\NormalTok{) }\SpecialCharTok{\&}\NormalTok{ (}\DecValTok{6} \SpecialCharTok{\textless{}} \DecValTok{9}\NormalTok{)}
\end{Highlighting}
\end{Shaded}

\begin{verbatim}
## [1] FALSE
\end{verbatim}

\begin{Shaded}
\begin{Highlighting}[]
\NormalTok{(}\DecValTok{4} \SpecialCharTok{\textgreater{}} \DecValTok{5}\NormalTok{) }\SpecialCharTok{|}\NormalTok{ (}\DecValTok{6} \SpecialCharTok{\textless{}} \DecValTok{9}\NormalTok{)}
\end{Highlighting}
\end{Shaded}

\begin{verbatim}
## [1] TRUE
\end{verbatim}

\begin{Shaded}
\begin{Highlighting}[]
\NormalTok{((}\DecValTok{4} \SpecialCharTok{\textgreater{}} \DecValTok{5}\NormalTok{) }\SpecialCharTok{\&}\NormalTok{ (}\DecValTok{6} \SpecialCharTok{\textless{}} \DecValTok{9}\NormalTok{)) }\SpecialCharTok{|}\NormalTok{ ((}\DecValTok{4} \SpecialCharTok{\textgreater{}} \DecValTok{5}\NormalTok{) }\SpecialCharTok{|}\NormalTok{ (}\DecValTok{6} \SpecialCharTok{\textless{}} \DecValTok{9}\NormalTok{))}
\end{Highlighting}
\end{Shaded}

\begin{verbatim}
## [1] TRUE
\end{verbatim}

\begin{Shaded}
\begin{Highlighting}[]
\NormalTok{((}\DecValTok{4} \SpecialCharTok{\textgreater{}} \DecValTok{5}\NormalTok{) }\SpecialCharTok{\&}\NormalTok{ (}\DecValTok{6} \SpecialCharTok{\textless{}} \DecValTok{9}\NormalTok{)) }\SpecialCharTok{\&}\NormalTok{ ((}\DecValTok{4} \SpecialCharTok{\textgreater{}} \DecValTok{5}\NormalTok{) }\SpecialCharTok{|}\NormalTok{ (}\DecValTok{6} \SpecialCharTok{\textless{}} \DecValTok{9}\NormalTok{))}
\end{Highlighting}
\end{Shaded}

\begin{verbatim}
## [1] FALSE
\end{verbatim}

\hypertarget{other-operators}{%
\subsection{Other Operators}\label{other-operators}}

\begin{itemize}
\tightlist
\item
  Create series or sequences of numbers\texttt{(:)}\\
\item
  Find out if an element belongs to a vector\texttt{(\%in\%)}
\end{itemize}

\begin{Shaded}
\begin{Highlighting}[]
\DecValTok{4} \SpecialCharTok{\textless{}} \DecValTok{5}
\end{Highlighting}
\end{Shaded}

\begin{verbatim}
## [1] TRUE
\end{verbatim}

\begin{Shaded}
\begin{Highlighting}[]
\DecValTok{5} \SpecialCharTok{==} \DecValTok{5}
\end{Highlighting}
\end{Shaded}

\begin{verbatim}
## [1] TRUE
\end{verbatim}

\begin{Shaded}
\begin{Highlighting}[]
\NormalTok{seq }\OtherTok{=} \DecValTok{1}\SpecialCharTok{:}\DecValTok{20}
\NormalTok{seq}
\end{Highlighting}
\end{Shaded}

\begin{verbatim}
##  [1]  1  2  3  4  5  6  7  8  9 10 11 12 13 14 15 16 17 18 19 20
\end{verbatim}

\begin{Shaded}
\begin{Highlighting}[]
\NormalTok{age }\OtherTok{\textless{}{-}} \DecValTok{20}
\NormalTok{age }\SpecialCharTok{\textless{}} \DecValTok{50}
\end{Highlighting}
\end{Shaded}

\begin{verbatim}
## [1] TRUE
\end{verbatim}

\begin{Shaded}
\begin{Highlighting}[]
\NormalTok{vowel\_sounds }\OtherTok{\textless{}{-}} \FunctionTok{c}\NormalTok{(}\StringTok{"a"}\NormalTok{,}\StringTok{"e"}\NormalTok{,}\StringTok{"i"}\NormalTok{,}\StringTok{"o"}\NormalTok{,}\StringTok{"u"}\NormalTok{)}
\StringTok{"a"} \SpecialCharTok{==}\NormalTok{ vowel\_sounds}
\end{Highlighting}
\end{Shaded}

\begin{verbatim}
## [1]  TRUE FALSE FALSE FALSE FALSE
\end{verbatim}

\begin{Shaded}
\begin{Highlighting}[]
\StringTok{"c"} \SpecialCharTok{\%in\%}\NormalTok{ vowel\_sounds}
\end{Highlighting}
\end{Shaded}

\begin{verbatim}
## [1] FALSE
\end{verbatim}

\hypertarget{data-types}{%
\section{Data Types}\label{data-types}}

\begin{itemize}
\tightlist
\item
  In R Programming variables can be stored as different data types.\\
\item
  These data types are not declared and automatically set to the
  variables depending on the type of values assigned to it.\\
\item
  In R there are 5 major data types, numeric integer complex character
  logical
\item
  In this workshop, we will treat just numeric, integer, character and
  logical. The rest are beyond the scope of this class.
  numeric\texttt{(1.3,2.4,5)}
\end{itemize}

\begin{Shaded}
\begin{Highlighting}[]
\NormalTok{a }\OtherTok{=} \DecValTok{45}
\NormalTok{b }\OtherTok{=} \FloatTok{4.3}
\NormalTok{a}
\end{Highlighting}
\end{Shaded}

\begin{verbatim}
## [1] 45
\end{verbatim}

\begin{Shaded}
\begin{Highlighting}[]
\NormalTok{b}
\end{Highlighting}
\end{Shaded}

\begin{verbatim}
## [1] 4.3
\end{verbatim}

integer\texttt{(2L,5L,6L)}

\begin{Shaded}
\begin{Highlighting}[]
\NormalTok{c }\OtherTok{=}\NormalTok{ 3L}
\NormalTok{d }\OtherTok{=}\NormalTok{ 56L}
\NormalTok{e }\OtherTok{=} \FloatTok{45.5}\NormalTok{L }\CommentTok{\#not an integer and will return error because it contains decimal }
\end{Highlighting}
\end{Shaded}

character\texttt{("apple","boy")}

\begin{Shaded}
\begin{Highlighting}[]
\NormalTok{name }\OtherTok{=} \StringTok{"Joy"}
\NormalTok{sex }\OtherTok{=} \StringTok{"Male"}
\end{Highlighting}
\end{Shaded}

logical\texttt{(TRUE\ or\ FALSE)}

\begin{Shaded}
\begin{Highlighting}[]
\NormalTok{x }\OtherTok{=} \ConstantTok{TRUE}
\NormalTok{y }\OtherTok{=}\NormalTok{ (}\DecValTok{5}\SpecialCharTok{\textless{}}\DecValTok{1}\NormalTok{)}
\end{Highlighting}
\end{Shaded}

\begin{itemize}
\tightlist
\item
  class of a data type can be checked using the class function
\end{itemize}

\begin{Shaded}
\begin{Highlighting}[]
\FunctionTok{class}\NormalTok{(a)}
\end{Highlighting}
\end{Shaded}

\begin{verbatim}
## [1] "numeric"
\end{verbatim}

\begin{Shaded}
\begin{Highlighting}[]
\FunctionTok{class}\NormalTok{(b)}
\end{Highlighting}
\end{Shaded}

\begin{verbatim}
## [1] "numeric"
\end{verbatim}

\begin{Shaded}
\begin{Highlighting}[]
\FunctionTok{class}\NormalTok{(c)}
\end{Highlighting}
\end{Shaded}

\begin{verbatim}
## [1] "integer"
\end{verbatim}

\begin{Shaded}
\begin{Highlighting}[]
\FunctionTok{class}\NormalTok{(d)}
\end{Highlighting}
\end{Shaded}

\begin{verbatim}
## [1] "integer"
\end{verbatim}

\begin{Shaded}
\begin{Highlighting}[]
\FunctionTok{class}\NormalTok{(e)}
\end{Highlighting}
\end{Shaded}

\begin{verbatim}
## [1] "numeric"
\end{verbatim}

\begin{Shaded}
\begin{Highlighting}[]
\FunctionTok{class}\NormalTok{(x)}
\end{Highlighting}
\end{Shaded}

\begin{verbatim}
## [1] "logical"
\end{verbatim}

\begin{Shaded}
\begin{Highlighting}[]
\FunctionTok{class}\NormalTok{(y)}
\end{Highlighting}
\end{Shaded}

\begin{verbatim}
## [1] "logical"
\end{verbatim}

\hypertarget{data-structures}{%
\section{Data Structures}\label{data-structures}}

\begin{itemize}
\tightlist
\item
  A data structure is a collection of data types, these data types can
  be similar or different from each other.\\
\item
  A data is divided into 6 major types;\\
  Vectors\\
  Lists\\
  Matrices\\
  Arrays\\
  Data Frames\\
  Factors
\end{itemize}

\hypertarget{vectors}{%
\subsection{Vectors}\label{vectors}}

\begin{itemize}
\tightlist
\item
  Vectors contain items or values of the same data type.\\
\item
  These values are usually combined with the \texttt{c()} function
  separated by \texttt{,}.
\end{itemize}

\begin{Shaded}
\begin{Highlighting}[]
\NormalTok{colours }\OtherTok{\textless{}{-}} \FunctionTok{c}\NormalTok{(}\StringTok{"blue"}\NormalTok{,}\StringTok{"green"}\NormalTok{,}\StringTok{"yellow"}\NormalTok{,}\StringTok{"black"}\NormalTok{,}\StringTok{"white"}\NormalTok{)}
\NormalTok{prime\_number }\OtherTok{\textless{}{-}} \FunctionTok{c}\NormalTok{(}\DecValTok{1}\NormalTok{,}\DecValTok{3}\NormalTok{,}\DecValTok{5}\NormalTok{,}\DecValTok{7}\NormalTok{,}\DecValTok{11}\NormalTok{,}\DecValTok{13}\NormalTok{)}
\end{Highlighting}
\end{Shaded}

\hypertarget{lists}{%
\subsection{Lists}\label{lists}}

\hypertarget{matrices}{%
\subsection{Matrices}\label{matrices}}

\hypertarget{arrays}{%
\subsection{Arrays}\label{arrays}}

\hypertarget{data-frames}{%
\subsection{Data Frames}\label{data-frames}}

\hypertarget{factors}{%
\subsection{Factors}\label{factors}}

\hypertarget{importing-and-exporting-data}{%
\section{Importing and Exporting
Data}\label{importing-and-exporting-data}}

\hypertarget{cleaning-data}{%
\section{Cleaning Data}\label{cleaning-data}}

\hypertarget{analyzing-and-visualizing-data}{%
\section{Analyzing and Visualizing
data}\label{analyzing-and-visualizing-data}}

\hypertarget{reporting-in-r}{%
\section{Reporting in R}\label{reporting-in-r}}

\end{document}
